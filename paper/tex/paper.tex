\documentclass[conference,compsoc,final,a4paper]{IEEEtran}
\usepackage[utf8]{inputenx}

%% Bitte legen Sie hier den Titel und den Autor der Arbeit fest
\newcommand{\autoren}[0]{Mulyadhi, Candra \and Mulyadi}
\newcommand{\dokumententitel}[0]{automatic navigation of a EV3-Robot through a maze using image recognition and machine learning}

\input{preambel} % Weitere Einstellungen aus einer anderen Datei lesen

\begin{document}

% Titel des Dokuments
\title{\dokumententitel}

% Namen der Autoren
\author{
    Mulyadhi, Candra \\
    Hochschule Mannheim\\
    Fakultät für Informatik\\
    Paul-Wittsack-Str. 10,
    68163 Mannheim
    \and
    Lieske, Jonathan \\
    Hochschule Mannheim\\
    Fakultät für Informatik\\
    Paul-Wittsack-Str. 10,
    68163 Mannheim

}

% Titel erzeugen
\maketitle
\thispagestyle{plain}
\pagestyle{plain}

% Eigentliches Dokument beginnt hier
% ----------------------------------------------------------------------------------------------------------

% Kurze Zusammenfassung des Dokuments
\begin{abstract}

\end{abstract}

% Inhaltsverzeichnis erzeugen
{\small\tableofcontents}

% Abschnitte mit \section, Unterabschnitte mit \subsection und
% Unterunterabschnitte mit \subsubsection
% -------------------------------------------------------
\section{Introduction}\label{introduction}
Die Einleitung liefert eine generelle Darstellung des Problems, der Ziele der Arbeit und deren Aufbau. Beschreibt den Hintergrund der Arbeit, das bearbeitete Problem und die Untersuchungsmethoden. Am Ende wird kurz der Aufbau der Arbeit erläutert.

Die Einleitung schreibt man erst, nachdem der Hauptteil der Arbeit fertig ist.

% -------------------------------------------------------
\section{Architecture}\label{architecture}

% class diagram

\subsection{Detector}

\subsection{Solver}

\subsection{Controller}

\subsection{Visualizer}

\subsection{Media}
input image of maze, ..


% -------------------------------------------------------
\section{Detecting shape and form of maze}\label{detection}

\subsection{}

% -------------------------------------------------------
\section{Finding the shortes path through the maze}\label{solving}

\subsection{}

% -------------------------------------------------------
\section{Establishing Bluetooth connection with EV3-Robot and controlling it}\label{solving}

\subsection{installations}
\subsection{connections}
\subsection{programming controls}


% --------------------------------------------------------------------
\section*{Abbreviations}
\addcontentsline{toc}{section}{Abbreviations}

% Die längste Abkürzung wird in die eckigen Klammern
% bei \begin{acronym} geschrieben, um einen hässlichen Umbruch zu verhindern
% Sie müssen die Abkürzungen selbst alphabetisch sortieren!
\begin{acronym}[IEEE]
\acro{A2A}{Application-to-Application}
\acro{ABK}{Abkürzung}
\acro{ACL}{Acess Control List}
\acro{ACM}{Association of Computing Machinery}
\acro{AES}{Advanced Encryption Standard}
\acro{IEEE}{Institute of Electrical and Electronics Engineers}
\acro{ISO}{International Organization for Standardization}
\acro{PDF}{Portable Document Format}
\end{acronym}

% Literaturverzeichnis
\addcontentsline{toc}{section}{Literature}
\printbibliography

\end{document}
